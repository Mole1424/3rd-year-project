\chapter{Design \& Implementation}
\label{ch:design-implementation}

\section{Language \& Libraries}

\begin{itemize}
    \item Python as pre-existing machine learning frameworks
    \item Tensorflow vs PyTorch (Tensorflow seemed nicer and more base code, first code seen for proof of concept was in tensorflow)
    \item mutually exclusive due to CUDA implementations
\end{itemize}

\section{Proof of Concept}

\begin{itemize}
    \item Re-use oriented design
    \item Quick and dirty
    \item Test my theory
    \item Explain methodologies
\end{itemize}

\section{Main Algorithm}

\begin{itemize}
    \item pseudocode of main algorithm
    \item if at any point something fails, mark as fake (ensures resiliency)
    \item emphasise each part is interchangeable
\end{itemize}

\subsection{Eye Landmark Detection}

\begin{itemize}
    \item EAR needs 6 points per eye so lets get them
    \item Facial landmarking is a common superset problem
    \item Introduce 68 landmarks and datasets
    \item Most popular ones aren't viable as built to be small to fit in a package (dlib, opencv...) or to run on a phone (mediapipe)
\end{itemize}

\subsubsection{Face cropping}

YuNet for first pass, MTCNN if yunet fails

\subsubsection{Weakly Supervised Eye Landmarks Detection (WSELD)}

\begin{itemize}
    \item Looked promising
    \item Most accurate according to landmarks in paper \& only model focussing specifically on eye landmarking
    \item Has a ``good enough" implementation details
    \item Had issues
    \begin{itemize}
        \item Landmarks from regions (mention other paper that managed it
        \item difficulty implementing in tensorflow
        \item after a month scrapped development
    \end{itemize}
\end{itemize}

\subsubsection{Pre-implemented Models}

\begin{itemize}
    \item Re-use oriented design
    \item papers with code
    \item Used best models which had pre-existing implementations (one for fast, one more accurate)
    \item couldve optimised for 6 landmarks but didnt want to mess with model architectures that were known working (batch size probably meant nothing would change anyway)
\end{itemize}

\subsubsection{HRNet}

\begin{itemize}
    \item modular sequences
    \item keeps high resolution layers in context at all times
    \item generic network but with a specific implementation for eye landmarking
    \item Heatmap per landmark
\end{itemize}

\subsubsection{PFLD}

\begin{itemize}
    \item speedy boi
    \item uses mobilenet-techniques to reduce the size and increase the speed
    \item still accurate tho
    \item secret sauce is custom loss function
\end{itemize}

\subsubsection{Datasets}

\begin{itemize}
    \item 68 landmarks (upsampling and downsampling where necessary)
    \item list all of them and give brief overview
    \item didnt use aflw due to inaccurate landmarks or not enough landmarks
    \item reflected to double size
\end{itemize}

\subsubsection{Final model choice}

Yunet + PFLD, then MTCNN and HRNET as backup (hope ear analysis can filter out idiosyncrasies of the model as will be consistent across real and fake)

\subsection{EAR Analysis}

\begin{itemize}
    \item Can be abstracted to a univariate time series
    \item Well researched
    \item Classical methods exists (pyts library)
    \item explain each one and give simple summary (1 paragraph)
    \item As do deep learning frameworks (the other papers)
    \item model diagrams for each one (maybe give a rough explenation as to the aims each one is doing?)
\end{itemize}

\section{Adversarial Noise}

\begin{itemize}
    \item foolbox library (Targeted FGSM using the other thing)
    \item chosen over cleverhans as cleverhans had meh documentation
    \item cw-l2 attack too slow (3hrs per video)
    \item need to still implement fakeretouch
\end{itemize}

\section{Final Code}

\begin{itemize}
    \item One codebase to do full training, splitting and evaluation
    \item just give overall flow diagram?
    \item {\huge models only trained on non-noisy images}
    \item checkpointing where possible
    \item run on either dcs compute clusters or avon
    \item final speeds + hardware
\end{itemize}